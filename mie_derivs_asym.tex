\documentclass[a4paper,10pt]{article}
\usepackage[utf8]{inputenc}
\usepackage{times}
\usepackage{amsmath,amssymb,stmaryrd}
\usepackage[colorlinks=true,menucolor=black,%
            citecolor=black,urlcolor=black,linkcolor=black,%
            anchorcolor=black,filecolor=black]{hyperref}
\usepackage{doi}
\usepackage{natbib}
\usepackage{graphicx}

\oddsidemargin  =   0.0mm
\textwidth      = 160.0mm


\def \Qsca { {Q_{\mbox{\rm sca}}}}
%\def \asym { {\langle \cos \theta \rangle} }
\def \asym { {g} }
\def \mied { \href{/home/crun/eodg/idl/mie/mie_derivs.pro}{\tt mie\_derivs.pro} } 
\def \mies { \href{/home/crun/eodg/idl/mie/mie_single.pro}{\tt mie\_single.pro} } 





\title{Derivatives of the asymmetry parameter in \mied}
\author{Andy Smith}



\begin{document}

\maketitle

\begin{abstract}
Following from \citet{Grainger2004}, here is the additional mathematics required to add the asymmetry parameter to the EODG \mied routine.
The parameters are included as keywords, so that the additional computational overhead is only carried out if required.
\end{abstract}


\section{Method}


Following the notation from \citet{Grainger2004}, the asymmetry parameter, $\asym$, is given by \citep{Bohren1983}:
\begin{equation}
 \Qsca \; \asym = \frac{4}{x^2} \left[   \sum_n^\infty \frac{n(n+2)}{n+1} \Re \left\{ a_na^*_{n+1} +  b_n b^*_{n+1}  \right\}
                                    +  \sum_n^\infty \frac{2n+1}{n(n+1)} \Re \left\{ a_n b^*_n \right\}\right].
\end{equation}



Taking the derivative wrt $x$, the size parameter, we obtain:
\begin{align}
 \frac{\partial\;}{\partial x} \left[\Qsca \; \asym \right] &= 
 \frac{4}{x^2}\sum_n^\infty \left[  \frac{n(n+2)}{n+1} \Re \left\{ 
                                                                \frac{\partial a_n}{\partial x} a^*_{n+1} + 
                                                                a_n \frac{\partial a^*_{n+1}}{\partial x}  +
                                                                \frac{\partial b_n}{\partial x} b^*_{n+1}  + 
                                                                b_n \frac{\partial b^*_{n+1}}{\partial x}
                                                       \right\} \right. \nonumber \\
                             & \;\;\;\;\;\;\;\;\;\;\;\; \;\;\;\;\;
                               \left.+ \frac{2n+1}{n(n+1)} \Re \left\{ a_n \frac{\partial b^*_n}{\partial x} + 
                                    \frac{\partial a_n}{\partial x} b^*_n  \right\}\right]  \nonumber\\
                      &\;\;\;\;\; - \frac{8}{x^3} \left[   \sum_n^\infty \frac{n(n+2)}{n+1} \Re \left\{ a_na^*_{n+1} +  b_n b^*_{n+1}  \right\}
                                    +  \sum_n^\infty \frac{2n+1}{n(n+1)} \Re \left\{ a_n b^*_n \right\}\right] \\
        &=\frac{4}{x^2}\sum_n^\infty \left[  \frac{n(n+2)}{n+1} \Re \left\{ 
                                                                \frac{\partial a_n}{\partial x} a^*_{n+1} + 
                                                                a_n \frac{\partial a^*_{n+1}}{\partial x}  +
                                                                \frac{\partial b_n}{\partial x} b^*_{n+1}  + 
                                                                b_n \frac{\partial b^*_{n+1}}{\partial x}
                                                       \right\} \right. \nonumber \\
                             & \;\;\;\;\;\;\;\;\;\;\;\; \;\;\;\;\;
                               \left.+ \frac{2n+1}{n(n+1)} \Re \left\{ a_n \frac{\partial b^*_n}{\partial x} + 
                                    \frac{\partial a_n}{\partial x} b^*_n  \right\}\right]  - \frac{2}{x} \; \asym \; \Qsca.
\end{align}
Defining:
\begin{align}
\xi_x=\sum_n^\infty \left[  \frac{n(n+2)}{n+1} \Re \left\{ 
                                                                \frac{\partial a_n}{\partial x} a^*_{n+1} + 
                                                                a_n \frac{\partial a^*_{n+1}}{\partial x}  +
                                                                \frac{\partial b_n}{\partial x} b^*_{n+1}  + 
                                                                b_n \frac{\partial b^*_{n+1}}{\partial x}
                                                       \right\}
                               + \frac{2n+1}{n(n+1)} \Re \left\{ a_n \frac{\partial b^*_n}{\partial x} + 
                                    \frac{\partial a_n}{\partial x} b^*_n  \right\}\right].
\end{align}
we say
\begin{align}
 \Qsca \; \frac{\partial \asym}{\partial x} + \frac{\partial \Qsca}{\partial x}\;\asym = \frac{4}{x^2} \xi_x - \frac{2}{x}\;\asym\;\Qsca,
\end{align}
leading to
\begin{align}
\frac{\partial \asym}{\partial x} = 
                 \frac{1}{\Qsca}\left[ \frac{4}{x^2} \xi_x - \asym\left(\frac{2\Qsca}{x} + \frac{\partial \Qsca}{\partial x}\right)\right].\label{eq:dgdx}
\end{align}




Similarly, for the real and imaginary parts of refractive index, $m=m_r + i m_i$, we define $\xi_r$ and $\xi_i$:
\begin{align}
 \xi_r &= \sum_n^\infty \left[  \frac{n(n+2)}{n+1} \Re \left\{ 
                                                                \frac{\partial a_n}{\partial m_r} a^*_{n+1} + 
                                                                a_n \frac{\partial a^*_{n+1}}{\partial m_r}  +
                                                                \frac{\partial b_n}{\partial m_r} b^*_{n+1}  + 
                                                                b_n \frac{\partial b^*_{n+1}}{\partial m_r}
                                                       \right\}
                               + \frac{2n+1}{n(n+1)} \Re \left\{ a_n \frac{\partial b^*_n}{\partial m_r} + 
                                    \frac{\partial a_n}{\partial m_r} b^*_n  \right\}\right];\\
 \xi_i &= \sum_n^\infty \left[  \frac{n(n+2)}{n+1} \Re \left\{ 
                                                                \frac{\partial a_n}{\partial m_i} a^*_{n+1} + 
                                                                a_n \frac{\partial a^*_{n+1}}{\partial m_i}  +
                                                                \frac{\partial b_n}{\partial m_i} b^*_{n+1}  + 
                                                                b_n \frac{\partial b^*_{n+1}}{\partial m_i}
                                                       \right\}
                               + \frac{2n+1}{n(n+1)} \Re \left\{ a_n \frac{\partial b^*_n}{\partial m_i} + 
                                    \frac{\partial a_n}{\partial m_i} b^*_n  \right\}\right],
\end{align}
and obtain the derivatives by:
\begin{align}
 \frac{\partial \asym}{\partial m_r} &= \frac{1}{\Qsca} \left[ \frac{4}{x^2} \xi_r - \asym \frac{\partial \Qsca}{\partial m_r} \right];\label{eq:dgdn}\\
 \frac{\partial \asym}{\partial m_i} &= \frac{1}{\Qsca} \left[ \frac{4}{x^2} \xi_i - \asym \frac{\partial \Qsca}{\partial m_i} \right]. \label{eq:dgdk}
\end{align}

During the execution of \mied, the calculations of $\xi_x$, $\xi_r$, and $\xi_i$ are carried out iteratively as we ascend through values of n.
At the end of the code, the final values are calculated using equations \ref{eq:dgdx},~\ref{eq:dgdn}, and \ref{eq:dgdk}.

\section{Tests}

A small number of tests have been carried out and appear to be fine. The calculations of $\asym$ agree with those from \mies, and derivatives are sensible when plotted over variations in $x$, $m_r$, and $m_i$, as shown in Fig.~\ref{fig:test}.

\begin{figure}
 \includegraphics[width=\textwidth]{/home/jupiter/eodg/smithan/idl/tinker/test_asym_mie_derivs.pdf}
 \caption{Tests of the derivatives calculated using \mied.}
 \label{fig:test}
\end{figure}




\begin{thebibliography}{2}
\expandafter\ifx\csname natexlab\endcsname\relax\def\natexlab#1{#1}\fi
\expandafter\ifx\csname url\endcsname\relax
  \def\url#1{{\tt #1}}\fi
\expandafter\ifx\csname urlprefix\endcsname\relax\def\urlprefix{URL }\fi
\expandafter\ifx\csname doiprefix\endcsname\relax\def\doiprefix{doi:}\fi
\bibitem[{Bohren and Huffman(1983)}]{Bohren1983}
Bohren, C.~F. and D.~R. Huffman, 1983: {\it Absorption and Scattering of Light
  by Small Particles\/}. Wiley-{VCH}, \doi{10.1002/9783527618156}.
\bibitem[{Grainger et~al.(2004)Grainger, Lucas, Thomas, and
  Ewen}]{Grainger2004}
Grainger, R.~G., J.~Lucas, G.~E. Thomas, and G.~B. Ewen, 2004: Calculation of
  {M}ie derivatives. {\it Applied Optics\/}, {\bf 43}(28):5386--5393,
  \doi{10.1364/AO.43.005386}.
\end{thebibliography}

\end{document}
